\documentclass{article}

\usepackage[T1]{fontenc}
\usepackage[utf8]{inputenc}
\usepackage{babel}

\usepackage{microtype}
\usepackage[hidelinks]{hyperref}

\usepackage{fancyqr}

\usepackage[tex]{listings}
\lstset{basicstyle=\ttfamily,breaklines,texcsstyle=*\bfseries,moretexcs=[1]{fancyqr,qrcode,FancyQrDoNotPrintSquare,FancyQrRoundCut,FancyQrHardCut,FancyQrLoad,FancyLoadDefault,fancyqrset}}
\lstMakeShortInline[language={[LaTeX]TeX}]|


\def\TikZ{Ti\textit{k}Z}
\def\tikzqr{\TikZ qr}

\title{The \texorpdfstring{\tikzqr}{fancyqr} package}
\author{%
	\texorpdfstring{Florian Sihler\\*
		\url{https://github.com/EagleoutIce/fancyqr}
	}{Florian Sihler}}
\date{Version v0.0 \textendash\ 2022/08/18}


\begin{document}
   \maketitle


	\tikzqr\ is a simple package to create fancy qr codes with the help of the \href{https://www.ctan.org/pkg/qrcode}{qrcode}-package.
	You may use |\fancyqr| just like the normal |\qrcode| (|\fancyqr[<qr-options>]{<url>}|).

	If you do want to hide a center square (e.g, because you want to embed an image) you can use |\FancyQrDoNotPrintSquare{<x>}{<y>}| to hide a rectangle with radius x and y set from the center. If you choose this option, the default |\FancyQrRoundCut| that rounds cut corners can be changed with |\FancyQrHardCut|.
	At the moment, there are six other styles |flat|, |frame|, |blob|, |glitch|, |swift|, and |dots|, that you can load (locally) by using |\FancyQrLoad{<name>}|. The default style is named |default| and can be 'reset' by |\FancyQrLoad{default}| or |\FancyLoadDefault|.

	There are the following extra qr-options (you can set all of them with |\fancyqrset{<keys>}|):
	% | Option            | Type    | Default  | Explanation                                                |
	% | ----------------- | ------- | :------: | ---------------------------------------------------------- |
	% | `image`           | LaTeX   |          | Automatically canter an image.[^1]                         |
	% | `image padding`   | number  |          | Additionally hide blocks (x & y) around the image.         |
	% | `image x padding` | number  |   `0`    | Additionally hide blocks (x) around the image.             |
	% | `image y padding` | number  |   `0`    | Additionally hide blocks (y) around the image.             |
	% | `gradient`        | boolean |   true   | Toggle the color gradient                                  |
	% | `color`           | color   |          | Disables the `gradient` and sets the qr color accordingly. |
	% | `l color`         | color   | `purple` | Set the top left gradient color.                           |
	% | `left color`      | color   |          | Alias for `l color`.                                       |
	% | `r color`         | color   |  `teal`  | Set the bottom right gradient color.                       |
	% | `right color`     | color   |          | Alias for `r color`.                                       |
	% | `gradient angle`  | angle   |  `135`   | Change the gradient angle.                                 |

	% The defaults are set like this:

	% ```LateX
	% \fancyqrset{image padding=0,gradient=true,gradient angle=135,r color=teal,l color=purple}
	% ```

	% [^1]: The package will automatically calculate the required `\FancyQrDoNotPrintSquare` (you have to make sure the, the qr code still has enough information to be readable). Therefore, the image will not scale with the qr code.

\end{document}